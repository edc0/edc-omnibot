% !TEX root = main.tex
% !TEX program = pdflatex


\section{Especificação e montagem do protótipo}
\label{sec:montagem}

SEÇÃO EM CONSTRUÇÃO

Para o projeto, foi realizada a compra dos componentes no Apêndice A (COMO CITAR APENDICES??)

Apesar da falta de redundância, robôs omnidirecionais com 3 rodas são utilizados com mais frequência por serem mais simples de se implementar, apresentarem custo mais baixo, e

Foram escolhidas rodas suecas de 90 graus para permitir o uso de apenas 3 rodas, gerando economia de atuadores, peso reduzido, menor custo total, etc.

motores:
https://www.banggood.com/6V-210RPM-Encoder-Motor-DC-Gear-Motor-with-Mounting-Bracket-and-Wheel-p-1044064.html?p=970719369296201312SG

\cite{siegwart2011introduction}, pg 97: sensores.

\section{Desenvolvimento Teórico}
\label{sec:teorico}

%% MODELAGEM:
%PARK: pg 468

Coordenadas de referência na Equação \ref{eq:world_ref} (\cite{siegwart2011introduction}), também utilizado por \cite{ritter2016modelagem}:
\begin{equation}
  \begin{pmatrix}
    x \\
    y \\
    \theta
  \end{pmatrix}
  =
  \begin{pmatrix}
    cos \theta & sen \theta & 0 \\
    -sen\theta & cos \theta & 0 \\
    0          & 0          & 1
  \end{pmatrix}
  \begin{pmatrix}
    x_W \\
    y_W \\
    \theta
  \end{pmatrix}
  \label{eq:world_ref}
\end{equation}

Cinemática direta, para um robô com 3 rodas dispostas em simetria radial em torno do centro da estrutura, é dada pela Equação \ref{eq:dk}. Diversos autores utilizam variações da mesma modelagem (\cite{rojas2006holonomic}, \cite{ritter2016modelagem}, \cite{pin1994new}, entre outros).

\begin{equation}
  \begin{pmatrix}
    v_x \\
    v_y \\
    \omega
  \end{pmatrix}
  =
  \frac{r}{3R}
  \begin{pmatrix}
    -\frac{3R}{\sqrt{3}} & 0   & \frac{3R}{\sqrt{3}} \\
    R                    & -2R & R                   \\
    1                    & 1   & R
  \end{pmatrix}
  \begin{pmatrix}
    \dot{\phi_1} \\
    \dot{\phi_2} \\
    \dot{\phi_3}
  \end{pmatrix}.
  \label{eq:dk}
\end{equation}

Cinemática inversa, obtida realizando a inversão da matriz 3x3, é dada pela Equação \ref{eq:ik}. Nota-se que esta inversão é simplificada no caso do robô com 3 rodas, visto que quando há mais rodas é formada uma matriz 3x$n$, e se deve realizar a pseudo-inversa, conforme demonstrado por \cite{rojas2006holonomic}. Nas equações apresentadas, $r$ é o raio de cada roda e $R$ o raio do robô (a distância do centro da roda ao centro da estrutura do robô).

\begin{equation}
  \begin{pmatrix}
    \dot{\phi_1} \\
    \dot{\phi_2} \\
    \dot{\phi_3}
  \end{pmatrix}.
  =
  \frac{1}{r}
  \begin{pmatrix}
    -\frac{\sqrt(3)}{2} & \frac{1}{2} & R \\
    0                   & -1          & R \\
    \frac{\sqrt(3)}{2}  & \frac{1}{2} & R
  \end{pmatrix}
  \begin{pmatrix}
    v_x \\
    v_y \\
    \omega
  \end{pmatrix}
  \label{eq:ik}
\end{equation}

Como pela classificação de \cite{campion1996structural} um \acrshort{tomr} é caracterizado na categoria (3,0), o modelo cinemático das equações \ref{eq:dk} e \ref{eq:ik} é controlável, estável e descreve a posição, orientação e suas derivadas de forma suficiente. % O modelo cinemático da Equação \ref{eq:dk} também é utilizado por \cite{rojas2006holonomic} e \cite{ritter2016modelagem}.

%% ODOMETRIA:
\cite{samani2007comprehensive}: Descrevem fórmulas para odometria, e dizem que se isso não for muito bom, nem adianta ter um controlador massa. Utilizam três rodas passivas com os encoders, para evitar problemas. Solução a meio enjambrada.

Para se realizar a odometria do robô com uma precisão melhor, \cite{lynch2017modern} ainda fornece as equações relacionadas ao \textbf{body twist} $v_b = (v_{bx}, v_{by}, \omega_b)^T$, que relaciona a velocidade do centro geométrico do robô com as velocidades de cada uma das rodas, que diferem caso o robô realize rotações durante a translação. Tal relação é dada pela Equação \ref{eq:twist}.

\begin{equation}
  v_b = No-entendi-nada.
  \label{eq:twist}
\end{equation}

Modelagens dinâmicas também podem ser utilizadas, relacionando o comportamento do robô não às velocidades das suas rodas, mas sim ao torque aplicado a cada uma pelos motores. No entanto, no caso do \acrshort{tomr}, se podem utilizar as velocidades das rodas como entradas, desde que haja um \emph{loop} de controle que garanta tais velocidades durante o acionamento (CITAR AS NOTAS DE AULA DO WALTER??? control.pdf).

%% PLANEJAMENTO DE TRAJETÓRIA:
\cite{lynch2017modern}, capítulo 9.

%% CONTROLE:
\cite{rojas2006holonomic}: Sugerem que utilizar um controlador para cada roda é melhor do que para cada grau de liberdade. No nosso caso, é tranquilo pois temos apenas 3 rodas, mantendo o mesmo número de controladores. Devido à realimentação externa lenta, utilizam um preditor no robô. Não entram em detalhes.
motores:
https://www.banggood.com/6V-210RPM-Encoder-Motor-DC-Gear-Motor-with-Mounting-Bracket-and-Wheel-p-1044064.html?p=970719369296201312SG

Given a desired trajectory q d (t), we can adopt the feedforward plus proportional feedback linear controller (11.1) of Chapter 11 to track the trajectory: \cite{lynch2017modern}

\cite{samani2007comprehensive}: Definir os coeficientes dos PIDs é uma novela, pois devemos levar em consideração parâmetros que possuem muita variação, como o coeficiente de atrito do solo, características das baterias, entre outros. Controle deles é bem legal.
%q̇ com (t) = q̇ d (t) + K p ( q̇ d (t) − q(t)),
%(13.11)
%where K p ∈ R 3×3 is positive definite and q(t) is an estimate of the actual con-
%figuration derived from sensors. Then q̇ com (t) can be converted to commanded
%wheel driving velocities u com (t) using Equation (13.7).


%: seguimento de trajetória: malha aberta (3.6.1); Feedback (3.6.2) do livro: It is very similar to the controllers presented in [39, 100]. Others can be found in [8, 52, 53, 137]. Controle com uma matriz de ganhos K para o espaço de estados. Estável e tal. Comentam sobre camadas: planejamento -> decisão -> controlador em tempo real -> hardware.



\cite{siegwart2011introduction}: 5.6.3.1 Introduction to Kalman filter theory

%A \textbf{modelagem cinemática} do \acrshort{tomr}, conforme desenvolvida por \cite{campion1996structural} e na forma utilizada por \cite{samani2007comprehensive}, é dada por
%\begin{equation}
%  \begin{pmatrix}
%    \dot{\phi_1} \\
%    \dot{\phi_2} \\
%    \dot{\phi_3}
%  \end{pmatrix}.
%  =
%  \frac{1}{r}
%  \begin{pmatrix}
%    -sen(\theta)               & cos(\theta)                & R \\
%    -sen(\frac{\pi}{3}-\theta) & -cos(\frac{\pi}{3}-\theta) & R \\
%    sen(\frac{\pi}{3}+\theta)  & -cos(\frac{\pi}{3}+\theta) & R
%  \end{pmatrix}
%  \begin{pmatrix}
%    v_x \\
%    v_y \\
%    \omega
%  \end{pmatrix}.
%  \label{eq:pkm}
%\end{equation}


\section{Implementação dos algoritmos}
\label{sec:software}

SEÇÃO EM CONSTRUÇÃO

\section{Avaliação experimental}
\label{sec:experimental}

SEÇÃO EM CONSTRUÇÃO
