\section{Especificação e montagem do protótipo}
\label{sec:montagem}

SEÇÃO EM CONSTRUÇÃO

Para o projeto, foi realizada a compra dos componentes no Apêndice A (COMO CITAR APENDICES??)

Apesar da falta de redundância, robôs omnidirecionais com 3 rodas são utilizados com mais frequência por serem mais simples de se implementar, apresentarem custo mais baixo, e

\cite{siegwart2011introduction}, pg 97: sensores.

\section{Desenvolvimento Teórico}
\label{sec:teorico}

SEÇÃO EM CONSTRUÇÃO

MODELAGEM: \cite{siegwart2011introduction}: considera omnidirecional uma coisa e holonômico outra. 3.2.1: dois sistemas de coordenadas, com 3 variáveis de interesse (x, y, $\theta$), e as matrizes de transformação de um para outro. Toda modelagem tá na pg 33.
\cite{spong2005robot} fala sobre o espaço de estados, pg 16 do pdf.
DEVE TER EM TORNO DE DEZ PÁGINAS

PARK: pg 468

Foram escolhidas rodas suecas de 90 graus para permitir o uso de apenas 3 rodas, gerando economia de atuadores, peso reduzido, menor custo total, etc.

 CONTROLE:
 \cite{rojas2006holonomic}: Sugerem que utilizar um controlador para cada roda é melhor do que para cada grau de liberdade. No nosso caso, é tranquilo pois temos apenas 3 rodas, mantendo o mesmo número de controladores. Devido à realimentação externa lenta, utilizam um preditor no robô. Não entram em detalhes.
motores:
https://www.banggood.com/6V-210RPM-Encoder-Motor-DC-Gear-Motor-with-Mounting-Bracket-and-Wheel-p-1044064.html?p=970719369296201312SGmaint.te

ODOMETRIA:
\cite{samani2007comprehensive}: Descrevem fórmulas para odometria, e dizem que se isso não for muito bom, nem adianta ter um controlador massa. Utilizam três rodas passivas com os encoders, para evitar problemas. Solução a meio enjambrada.

\cite{siegwart2011introduction}: 5.6.3.1 Introduction to Kalman filter theory

A \textbf{modelagem cinemática} do \acrshort{tomr}, conforme desenvolvida por \cite{campion1996structural} e na forma utilizada por \cite{samani2007comprehensive}, é dada por

\begin{equation}
  \begin{pmatrix}
    \dot{\phi_1} \\
    \dot{\phi_2} \\
    \dot{\phi_3}
  \end{pmatrix}.
  =
  \frac{1}{r}
  \begin{pmatrix}
    -sen(\theta)               & cos(\theta)                & R \\
    -sen(\frac{\pi}{3}-\theta) & -cos(\frac{\pi}{3}-\theta) & R \\
    sen(\frac{\pi}{3}+\theta)  & -cos(\frac{\pi}{3}+\theta) & R
  \end{pmatrix}
  \begin{pmatrix}
    v_x \\
    v_y \\
    \omega
  \end{pmatrix}.
  \label{eq:pkm}
\end{equation}

Diversos autores utilizam variações da mesma modelagem (\cite{rojas2006holonomic}, \cite{ritter2016modelagem}, \cite{pin1994new}, entre outros). \cite{ritter2016modelagem} ainda implementa dois sistemas de coordenadas: um absoluto, do mundo, e outro centrado no robô, ambos rotacionados em relação um ao outro por $\theta$. Na Equação \ref{eq:pkm}, quando $\theta = 0$, temos as cinemáticas direta e inversa que pode ser utilizada para o controle das juntas, dadas por

\begin{equation}
  \begin{pmatrix}
    \dot{\phi_1} \\
    \dot{\phi_2} \\
    \dot{\phi_3}
  \end{pmatrix}.
  =
  \frac{1}{r}
  \begin{pmatrix}
    -\frac{\sqrt(3)}{2} & \frac{1}{2} & R \\
    0                   & -1          & R \\
    \frac{\sqrt(3)}{2}  & \frac{1}{2} & R
  \end{pmatrix}
  \begin{pmatrix}
    v_x \\
    v_y \\
    \omega
  \end{pmatrix}
  \label{eq:ik}
\end{equation}

e
con
\begin{equation}
  \begin{pmatrix}
    v_x \\
    v_y \\
    \omega
  \end{pmatrix}
  =
  \frac{r}{3R}
  \begin{pmatrix}
    -\frac{3R}{\sqrt{3}} & 0   & \frac{3R}{\sqrt{3}} \\
    R                    & -2R & R                   \\
    1                    & 1   & R
  \end{pmatrix}
  \begin{pmatrix}
    \dot{\phi_1} \\
    \dot{\phi_2} \\
    \dot{\phi_3}
  \end{pmatrix}.
  \label{eq:dk}
\end{equation}


Como pela classificação de \cite{campion1996structural} um \acrshort{tomr} é caracterizado na categoria (3,0), o modelo cinemático da Equação \ref{eq:pkm} é controlável, estável e descreve a posição, orientação e suas derivadas de forma suficiente. O modelo cinemático da Equação \ref{eq:pkm} também é utilizado por \cite{rojas2006holonomic} e \cite{ritter2016modelagem}, \cite{loh2003mechatronics} mostra que um \acrshort{tomr} não possui singularidades a partir da mesma modelagem (VERIFICAR).

\textbf{Modelagens dinâmicas} também podem ser utilizadas, relacionando o comportamento do robô não às velocidades das suas rodas, mas sim ao torque aplicado a cada uma pelos motores. No entanto, no caso do \acrshort{tomr}, se podem utilizar as velocidades das rodas como entradas, desde que haja um \emph{loop} de controle que garanta tais velocidades durante o acionamento (CITAR AS NOTAS DE AULA DO WALTER??? control.pdf).

\section{Implementação dos algoritmos}
\label{sec:software}

SEÇÃO EM CONSTRUÇÃO

\section{Avaliação experimental}
\label{sec:experimental}

SEÇÃO EM CONSTRUÇÃO
