\section{Conclusão e Trabalhos Futuros}
\label{sec:conclusao}

%%%%%%% Conclusões específicas:

Diante dos resultados obtidos é possível concluir que o protótipo desenvolvido mostrou estar apto para o desenvolvimento do tipo de trabalho proposto. Como o desenvolvimento teórico e implementação de \textit{softwares} são relativamente flexíveis e não envolvem nenhum custo de material, a utilização de uma plataforma pronta em trabalhos futuros é bastante desejável.

A modelagem cinemática desenvolvida para o robô se mostrou adequada, sendo possível converter os valores de velocidade desejados em termos de coordenadas globais em velocidades das rodas de maneira bastante adequada.

Os motores utilizados apresentam uma faixa relativamente restrita de velocidades, sendo que os efeitos de atrito da caixa de redução afeta os limites mais baixos e a própria relação de redução aliada à tensão nominal do robô limita a velocidade máxima. Os limites mínimos podem ser melhorados por meio de técnicas de controle melhores do que a utilizada. Para aplicações reais, se estima que a velocidade máxima esteja adequada na maioria das situações domésticas e industriais. Uma exceção importante seria, por exemplo, o uso do protótipo para futebol de robôs, onde a agilidade é valorizada. Neste caso, também seria necessário um método para decodificação mais rápida dos \textit{encoders}, aumento da taxa de amostragem do controlador e um possível estudo sobre a derrapagem das rodas.

O controle de velocidade das rodas apresenta limitações, devido à sua implementação simples. No entanto, para a maioria das aplicações e situações experimentadas, apresentou resultados satisfatórios. As rodas apresentaram deslizamento em relação ao chão apenas em situações extremas (reversão instantânea sobre chão empoeirado).

A instrumentação, para fins de controle de velocidade na faixa de velocidades utilizada, se mostrou suficiente. Para realização das integrações necessárias à odometria, no entanto, se deve investigar melhor os resultados obtidos.

Foram realizados seguimentos de trajetória simples, em malha aberta, sem contar com dados da odometria, apresentando alguns desvios significativos. Em trajetos curtos (menores do que 1 m), tais desvios poderiam ser desprezados em aplicações que não necessitem precisão. No entanto, como os desvios se propagam conforme a trajetória aumenta, se mostrou que um sistema de controle de posição é essencial para utilização prolongada.

%%%%%%%%% Conclusões gerais:

De um modo geral, se percebeu a forte interdependência entre todas as partes que compoem um sistema robótico. Se compreendeu que para o desenvolvimento de um projeto em robótica, é necessário que o engenheiro tenha um conhecimento profundo de todas as partes, isoladamente, e que saiba prever a influência de um componente na integração do todo. No caso de um trabalho em equipe, onde cada membro seja responsável por (e especialista em) um tipo específico de subsistema, ainda assim é necessário entender a influência da sua área nas demais, reforçando o caráter interdisciplinar da área e a importância de uma boa comunicação entre as partes.

Dentre os tipos de subsistema a serem projetados, se destaca a importância de uma boa seleção de hardware, visto que esta parte é bem menos flexível do que o desenvolvimento dos algoritmos de acionamento, por exemplo. No caso em que se deseja alguma modificação de software, basta reprogramar o sistema. Mesmo que tal ato envolva mão-de-obra especializada e tempo de trabalho, não envolve os custos de material, prazos de entrega, disponibilidade comercial e demais requisitos de um sistema físico.

%%%%%%%%%%%%%%TRABALHOS FUTUROS:

Se considera duas abordagens possíveis para trabalhos futuros relacionados ao tema. Se pode finalizar a implementação de um sistema completo, conforme o diagrama da Figura \ref{fig:sistema}, ou realizar esforços no sentido de aprimorar um subsistema específico. As duas abordagens são complementares.

Em se adotando uma abordagem mais geral, trabalhos futuros podem ser tocados no desenvolvimento um algoritmo de odometria mais preciso e um controle de posição capaz de seguir trajetórias em malha fechada. O trabalho de desenvolvimento de uma interface com o usuário também é importante e poderia ser focado.

Para melhorias específicas, futuros trabalhos podem ser sobre estudos relacionados aos seguintes tópicos:
\begin{itemize}
  \item{desenvolver uma modelagem dinâmica para o robô, considerando as massas e distribuição geométrica dos componentes utilizados;}
  \item{aprimorar o controle de velocidade para operação em velocidades baixas;}
  \item{terminar a implementação dos sensores inerciais e da bússola, aplicando algum algoritmo de fusão de sensores a esses dados e a odometria;}
  \item{implementação de uma biblioteca com diversas trajetórias pré computadas;}
  \item{realizar estudos sobre o desempenho real do computador utilizado, e otimizar a operação do código para ciclos de execução mais rápidos. Alternativas incluem a utilização de \textit{kernel} de tempo real, \textit{multithreading} e utilização da GPU do computador;}
  \item{quantificar o consumo de energia do robô, e implementar estratégias de operação mais eficientes;}
  \item{organizar e documentar o código-fonte seguindo boas práticas de programação, para agilizar o desenvolvimento de futuros trablahos utilizando os mesmos recursos;}
  \item{desenvolver aplicações práticas para o robô.}
\end{itemize}
