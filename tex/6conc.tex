\section{Conclusão e Trabalhos Futuros}
\label{sec:conclusao}

Terminar a implementação dos sensores inerciais e da bússola, aplicando algum algoritmo de fusão de sensores a esses dados e a odometria. Cabeamento de 4 vias para a rede I2C da bússola e da IMU.

O chassi podia ser menos espesso, e as peças poderiam ser melhor distribuídas, para utilizar uma área menor da placa. Assim, se poderia instalar mais sensores, ou simplesmente reduzir o tamanho do robô.

Modelagem dinâmica.

Motores melhores.

IMplementar mais trajetórias.

Verificar a temporização do programa, e otimizar. Talvez instalar algum kernel em tempo real, ou utilizar um microprocessador externamente ao comptador para executar os loops de controle.

Caso se deseje utilizar a plataforma projetada para futuros projetos e expansões, talvez seja uma boa ideia organizar melhor os códigos, documentações, verificar se alguma otimização pode ser realizada nas bibliotecas para tornar o software mais robusto, modular e de fácil manutenção. GIthub blaalba.

Láaaaaa no fim, melhorar a interface com o usuário, implementando aplicações práticas para o robô. Exemplos: n sei.

Sensores que utilizam a mesma tensão. Aumentar o número de reguladores ou achar um regulador mais potente para os motores (ou eu posso ligar direto na bateria???). POde ser bacana também adicionar mais uma bateria, exclusivamente para o acionamento dos motores (conforme utilizado em diversos trabalhos, como ACHAR UM TRABALHO).

Os motores representaram 38\% do custo total do projeto, e são o componente mais caro de todos. Fica o questionamento.

Processamento na GPU.


De um modo geral, se percebeu a forte interdependência entre todas as partes que compoem um sistema robótico. Se compreendeu que para o desenvolvimento de um projeto em robótica, é necessário que o engenheiro tenha um conhecimento profundo de todas as partes, isoladamente, e que saiba prever a influência de um componente na integração do todo. No caso de um trabalho em equipe, onde cada membro seja responsável por (e especialista em) um tipo específico de subsistema, ainda assim é necessário entender a influência da sua área nas demais, reforçando o caráter interdisciplinar da área e a importância de uma boa comunicação entre as partes.

Dentre os tipos de subsistema a serem projetados, se destacou a importância de uma boa seleção de hardware, visto que esta parte é bem menos flexível do que o desenvolvimento dos algoritmos de acionamento, por exemplo. No caso em que se deseja alguma modificação de software, basta reprogramar o sistema. Mesmo que tal ato envolva mão-de-obra especializada e tempo de trabalho, não envolve os custos de material, prazos de entrega, disponibilidade comercial e demais requisitos de um sistema físico.
