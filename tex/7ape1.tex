\subsection{Custo dos componentes utilizados}
\label{sec:custo}

Há no mercado uma variada gama de componentes a serem utilizados em projetos robóticos. Para o projeto em questão, se utilizaram componentes que mostrassem um preço de mercado competitivo e grande disponibilidade. Dessa forma, se pode manter o projeto viável, mesmo que por vezes sacrificando um possível incremento de desempenho que se daria ao utilizar um componente mais robusto, por exemplo.

Na Tabela \ref{tab:custo} se pode ver a lista de componentes adquirida e os respectivos custos. Nota-se que no caso das 3 rodas há -- integrado ao valor apresentado -- as taxas de importação e conversão de moedas, visto que esses componentes foram importados dos Estados Unidos. Também é importante mencionar que o chassi foi produzido sem custo BLABLABLAB LAMECC.

\begin{table}
  \caption{Custo dos componentes utilizados no projeto.}
  \begin{tabular}{||l r c r||}
     \hline
     \textbf{Item:}   & \textbf{Valor por unidade:} & \textbf{Quantidade:} & \textbf{Valor total:} \\ \hline\hline
     Omniwheel        & R\$ 46,76                   &  3                   & R\$ 140,29            \\ \hline
     Motor c/ encoder & R\$ 119,00                  &  3                   & R\$ 357,00            \\ \hline
     Driver           & R\$ 15,00                   &  2                   & R\$ 30,00             \\ \hline
     Raspberry Pi     & R\$ 148,00                  &  1                   & R\$ 148,00            \\ \hline
     microSD 16GB     & R\$ 34,00                   &  1                   & R\$ 34,00             \\ \hline
     Arduino Mega     & R\$ 40,00                   &  1                   & R\$ 40,00             \\ \hline
     IMU MPU6050      & R\$ 9,00                    &  1                   & R\$ 9,00              \\ \hline
     Magnetômetro HMC5883 & R\$ 12,80               &  1                   & R\$ 12,80             \\ \hline
     Chassi           & R\$ 5,00                    &  1                   & R\$ 5,00              \\ \hline
     Bateria          & R\$ 100,00                  &  1                   & R\$ 100,00            \\ \hline
     Reguladores de Tensão & R\$ 4,98               &  4                   & R\$ 19,96             \\ \hline
     Diversos         & R\$ 30,00                   &  X                   & R\$ 30,00             \\ \hline\hline
                      &                             & \textbf{Custo Total:} & R\$ 926,05           \\ \hline
  \end{tabular}
  \label{tab:custo}
\end{table}

Numa escala de preços em robótica, se percebe que (COMPARAR OS PREÇOS DE OUTROS PROTÓTIPOS??)
