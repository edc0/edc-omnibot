% !TEX root = main.tex
% !TEX program = pdflatex

\documentclass[portugues]{automatex}

\title{Desenvolvimento de plataforma robótica omnidirecional}
\shorttitle{Robô móvel omnidirecional}
\author{Emílio Dolgener Cantú}
\supervisor{Eduardo Perondi}

\newacronym{rpi}{RPi}{\emph{Raspberry Pi}, computador embarcado}
\newacronym{imu}{IMU}{\emph{Inertial Measurement Unit}, ou Unidade de Medidas Inerciais}
\newacronym{pid}{PID}{Controlador Proporcional-Integral-Derivativo}
\newacronym{pi}{PI}{Controlador Proporcional-Integral}
\newacronym{tomr}{TOMR}{\emph{Three-wheeld Omnidirectional Mobile Robot}, ou Robô Móvel Omnidirecional de Três Rodas}
\newacronym{ins}{INS}{\emph{Inertial Navigation System}, ou Sistema de Navegação Inercial}
\newacronym{pkm}{PKM}{\emph{Posture Kinematic Model}, ou Modelo Cinemático de Postura}
\newacronym{ckm}{CKM}{\emph{Configuration Kinematic Model}, ou Modelo Cinemático de Configuração}
\newacronym{cdm}{CDM}{\emph{Configuration Dynamic Model}, ou Modelo Dinâmico de Configuração}
\newacronym{pdm}{PDM}{\emph{Posture Dynamic Model}, ou Modelo Dinâmico de Postura}
\newacronym{i2c}{I2C}{\emph{Inter-Integrated Circuit}, ou Circuito Inter-Integrado}
\newacronym{arm}{ARM}{\emph{Advanced RISC Machine}}
\newacronym{gpio}{GPIO}{\emph{General Purpose Input/Output}}

% Símbolos:
\newglossaryentry{angvel}{
  name = $\omega$ ,
  description = Velocidade ângular,
}
\newglossaryentry{vx}{
  name = ${v_x}$ ,
  description = Velocidade linear no eixo X do sistema de coordenadas cartesiano,
}
\newglossaryentry{vy}{
  name = ${v_y}$,
  description = Velocidade linear no eixo Y do sistema de coordenadas cartesiano,
}
\newglossaryentry{deltam}{
  name = $\delta_m$,
  description = Grau de mobilidade,
}
\newglossaryentry{deltas}{
  name = $\delta_s$,
  description = Grau de dirigibilidade,
}
\newglossaryentry{velw}{
  name =$\phi_i$,
  description = Velocidade angular da roda \emph{i},
}
\newglossaryentry{raio}{
  name = $r$,
  description = Raio das rodas utilizadas,
}
\newglossaryentry{Raio}{
  name = $R$,
  description = Distância de cada roda até o centro geométrico da estrutura,
}

\agradecimentos{Agradeço a todo mundo que ajudou.}

\resumo{O presente trabalho apresenta a implementação de uma base robótica omnidirecional holonômica, utilizando 3 omniwheels. Foram implementados um sistema de controle e um método de odometria em malha aberta. Os resultados foram xxxx.}
\myabstract{This work shows the implementation of a holonomic omnidirectional robotic platform, using 3 omniwheels. Were also implemented a control system and a method for open loop odometry. As a result, xxxx.}

\begin{document}

\maketitle % TODO: tem como dispensar esse comando, tipo redefinindo o ambiente "document"

%%%%%%%%%%%%%%%%%%%%%%%%%%%%
%% Início do documento
%%%%%%%%%%%%%%%%%%%%%%%%%%%%

%% Trabalho principal
\section{Introdução}

%\subsection{Motivação}

O interesse na área da robótica se dá pela multi-disciplinaridade do tema, que abrange um espectro de conhecimentos que vai desde mecânica estrutural até a aplicação de teorias de controle sofisticadas. A área ainda se extende por eletrônica, elétrica, computação e até mesmo psicologia. Assim, considera-se que esta área é bastante adequada a um trabalho de conclusão de um curso igualmente abrangente, que é a Engenharia de Controle e Automação.

Em aplicações industriais, a maioria dos robôs utilizados são manipuladores, que realizam tarefas repetitivas -- como soldagem ou montagem de peças -- com precisão e rapidez adequados a cada aplicação. Estes robôs, no entanto, são em geral fixos, e daí surge o estudo da robótica móvel: como um robô pode se mover sem supervisão humana e interagir com o mundo real? \citep{siegwart2011introduction} Além do interesse acadêmico, existe um significativo interesse comercial, visto que o mercado de robôs móveis, que estava em torno de 4,5 bilhões de dólares americanos em 2014, tende a duplicar até 2020 \citep{marketsmarkets}.

\begin{figure}[h]
  \centering
  \includegraphics[width = 0.5\textwidth]{imagens/markets}
  \caption{Projeção de crescimento do mercado de robótica móvel até 2020.}
  \label{fig:markets}
  \source{\citet{marketsmarkets}}
\end{figure}

Segundo \citet{lynch2017modern}, robôs móveis são divididos em não-holonômicos e omnidirecionais, com diferenças significativas em planejamento de trajetória, controle e modelagem dos dois tipos de robô. Diante da natureza do presente trabalho de conclusão de curso, o escopo foi definido no âmbito dos robôs omnidirecionais, que se destacam pela habilidade de realizar transporte de cargas pequenas em espaços confinados -- como corredores de hospital e depósitos de armazenamento que buscam o aumento da capacidade sem perder agilidade logística nem aumentar o espaço necessário nas instalações. Academicamente, o controle de rodas omnidirecionais apresenta diversos problemas atrativos, vários dos quais serão descritos ao longo do presente trabalho. O desenvolvimento de uma plataforma robótica omnidirecional holonômica se torna útil para futuras aplicações em diversas áreas de investigação em robótica, controle e automação.

%\subsection{Descrição do Trabalho}

O presente trabalho consiste no desenvolvimento (tanto teórico quanto experimental) de uma plataforma robótica que possa se movimentar de maneira autônoma em qualquer direção do plano sem necessidade de reorientação -- apresentando holonomicidade. Após uma avaliação na bibliografia sobre os tipos de robôs factíveis de serem construídos no tempo previsto e com os recursos financeiros disponíveis, optou-se pela configuração descrita na sequência. A plataforma considerada mais adequada utiliza 3 \emph{omniwheels}, cada uma acionada por um motor elétrico dedicado, conforme o modelo mostrado na Figura \ref{fig:tomr_ritter}. Como as rodas são montadas de maneira fixa no chassi, este tipo de robô ainda oferece a vantagem de ser construído com uma estrutura mecânica mais simples, como mencionado por \citet{siciliano2016springer}. As rodas omnidirecionais utilizadas podem ser vistas em detalhess na Figura \ref{fig:omniwheel}. O sistema de movimentação é controlado por software processado em um computador embarcado a partir dos sinais fornecidos por sensores inerciais e de odometria.

\begin{figure}[h]
  \centering
  \includegraphics[width = 0.45\textwidth]{imagens/tomr_ritter_mod}
  \caption{Diagrama de um robô móvel com três rodas omnidirecionais.}
  \label{fig:tomr_ritter}
  \source{Adaptado de \citet{ritter2016modelagem}}
\end{figure}

\begin{figure}[h]
  \centering
  \includegraphics[width = 0.45\textwidth]{imagens/omniwheel}
  \caption{Desenho de uma roda omnidirecional.}
  \label{fig:omniwheel}
  \source{Adaptado de \citet{omniwheel}}
\end{figure}

%\subsection{Objetivos}

Este trabalho tem como \textbf{objetivo geral} projetar, construir, colocar em operação e testar uma plataforma robótica omnidirecional de baixo custo, mas com características semelhantes às dos sistemas comerciais. Para atingir esse objetivo, se devem alcançar os seguintes \textbf{objetivos específicos}:

\begin{itemize}
  \item{Modelagem do robô;} %especificar se é cinemática ou dinâmica
  \item{Especificação e construção de um protótipo;}
  \item{Implantação de um algoritmo de controle;} % eu tinha colocado implementação, mas perondi pediu implantação
  \item{Implantação de instrumentação e de algoritmo de localização;}
  \item{Realizar experimentos de seguimento de trajetórias e analisar os resultados obtidos.}
\end{itemize}

%\subsection{Organização do Trabalho}

O trabalho está organizado da seguinte maneira:
\begin{itemize}
  \item{Na \hyperref[sec:revbib]{Seção 2}, é apresentada a revisão bibliográfica, versando sobre robótica móvel, técnicas de controle utilizadas na área e um breve resumo sobre métodos de localização, além de uma exposição dos trabalhos mais recentes envolvendo robôs omnidirecionais;} %colocar planejamento de trajetória e hardware tbm??
  \item{Na \hyperref[sec:montagem]{Seção 3}, foca-se na especificação do hardware, estrutura mecânica e montagem do protótipo;}
  \item{Na \hyperref[sec:teorico]{Seção 4}, é apresentado o desenvolvimento teórico da modelagem do robô e dos algoritmos de controle e localização;}
  \item{Na \hyperref[sec:software]{Seção 5} são implementados os algoritmos descritos na seção anterior;}
  \item{Na \hyperref[sec:experimental]{Seção 6}, é realizado um experimento para avaliar o desempenho do robô projetado;}
  \item{Por fim, na \hyperref[sec:resultados]{Seção 7}, são apresentadas as análises e discussões sobre os resultados dos experimentos, a conclusão sobre o trabalho como um todo e algumas propostas para futuros trabalhos.}
\end{itemize}

Além do descrito, o trabalho ainda contém os seguintes apêndices:
\begin{itemize}
  \item{\hyperref[sec:custo]{Apêndice A}, da descrição dos custos do projeto;}
  \item{\hyperref[sec:loop]{Apêndice B}, com o código fonte da função executada periodicamente para o controle;}
  \item{\hyperref[sec:draw]{Apêndice C}, que mostra as dimensões da plataforma projetada.}
\end{itemize}

\section{Revisão Bibliográfica}
\label{sec:revbib}

\subsection{Fundamentação Teórica}

% Robôs móveis: ESTA PARTE ESTÁ NA INTRODUÇÃO
% - breve história
% - tipos
% - pq omnidirecionais?
% - tipos de modelagem

EM CONSTRUÇÃO

% centralizado vs descentralizado: 22 do indiveri
% intro do oubatti é muito boa
% samani tem uma bela revisão de controle

% Odometria e Localização
% - métodos absolutos
% - métodos relativos e dead reckoning
% - fusão de sensores, kalman

% Controle
% - controle centralizado
% - controle descentralizado
% - coisas mais novas?
% - geração de trajetória

%A \textbf{modelagem cinemática} do \acrshort{tomr}, conforme desenvolvida por \cite{campion1996structural} e na forma utilizada por \cite{samani2007comprehensive}, apresentada mais adiante. Também existem \textbf{modelagens dinâmicas} também podem ser utilizadas, relacionando o comportamento do robô não às velocidades das suas rodas, mas sim ao torque aplicado a cada uma pelos motores. No entanto, no caso do \acrshort{tomr}, se podem utilizar as velocidades das rodas como entradas, desde que haja um \emph{loop} de controle que garanta tais velocidades durante o acionamento (CITAR AS NOTAS DE AULA DO WALTER??? control.pdf).

% Hardware
% - arduino
% - raspberry pi
% - módulos de sensores de baixo custo
% - estrutura mecânica

%O \textbf{Raspberry Pi} é um \emph{single board computer}, que utiliza a arquitetura \acrshort{arm} em seu processador, ideal para dispositivos alimentados por baterias por consumir pouca energia e gerar pouco calor. O processador possui quatro núcleos, e um \emph{clock} de 1,2 GHz -- poder computacional equivalente há um computador de mesa comum. O \acrshort{rpi} utiliza um sistema operacional GNU/Linux, e \emph{software} deve ser desenvolvido para ser executado nesta plataforma. Há ainda 40 pinos de \acrshort{gpio} que podem ser utilizados para conectar sensores, atuadores e diversos componentes, e suporte nativo a \acrshort{i2c} (\cite{upton2014raspberry}).

% Software, implementação
% - tempo de ciclo
% - real-time
% - protocolos de comunicação

%Para a comunicação dos periféricos com este computador, é necessário utilizar algum protocolo de comunicação. O protocolo \textbf{\acrlong{i2c}}, é geralmente utilizado em robôs, com um grande suporte tanto pela \acrshort{rpi} (\cite{upton2014raspberry}) quanto pelos componentes em geral utilizados (\cite{MPU6050} e a bússola e o arduino se eu usar). Com este protocolo, descrito em \cite{semiconductors2000i2c}, dados podem ser transmitidos a 100 Kbps -- ou 400 Kbps quando utilizado o \emph{fast mode}. São utilizados duas linhas bidirecionais no barramento: SDA para os dados e SCL para os sinais de \emph{clock}. O número de dispositivos conectados ao barramento só depende do limite de capacitância descrito na especificação. Resistores de \emph{pull-up} são necessários para manter a linha em estado lógico alto quando não utilizada, porém estes resistores estão presentes internamente no \emph{Raspberry Pi}, por exemplo.

\subsection{Estado da Arte}

%% RODAS
A grande maioria dos robôs construídos com \textbf{\emph{omniwheels}} utiliza 3 rodas em uma configuração triangular simétrica -- como apresentado na Figura \ref{fig:tomr_ritter} --, à exemplo de \cite{ritter2016modelagem}, \cite{samani2007comprehensive}, \cite{williams2002dynamic} e \cite{indiveri2009swedish}, entre outros. Alguns autores, como \cite{krinkin2015design} e \cite{rojas2006holonomic} utilizam 4 rodas, este último desenvolvendo algoritmos para que o robô continue operando mesmo que um dos motores deixe de funcionar. Em diversos trabalhos surge uma preocupação em relação a possíveis derrapagens das rodas ao aumentar a velocidade de operação. \cite{williams2002dynamic} apresenta um estudo sobre os coeficientes de atrito de rodas omnidirecionais em diversas superfícies e um modelo dinâmico que leva tal efeito em consideração.

%% MODELAGEM e CONTROLE
Há suficientes técnicas de \textbf{controle} para que cada autor utilize a que mais convenha às suas necessidades. \cite{ritter2016modelagem} utiliza um PID para cada roda, com parâmetros escolhidos empiricamente. Por sua vez, o robô torna-se difícil de controlar com velocidades acima de 1 m/s, segundo suas simulações. \cite{samani2007comprehensive} utiliza 3 PIDs, para controlar posição e orientação do robô. Em contraste, \cite{rojas2006holonomic} e \cite{indiveri2009swedish} também utilizam PIDs, porém para o controle de cada individual de cada motor, utilizando o modelo cinemático apenas. \cite{indiveri2009swedish} também sugere táticas para evitar saturação dos atuadores.

Tanto \cite{treesatayapun2011discrete} e \cite{oubbati2005velocity} utilizam redes neurais para atribuir parâmetros aos controladores, sendo que no primeiro se tem uma estrutura de controle baseada em redes neurais enquanto no segundo se utilizam as redes para calcular os parâmetros de 5 PIDs, melhorando o desempenho em relação às não-linearidades dos modelos dinâmicos utilizados. \cite{oubbati2005velocity} ainda menciona que os resultados obtidos não foram tão bons quanto poderiam ser, devido à dificuldade de se coletar dados de treinamento para as redes.

%% ODOMETRIA:
A maioria das implementações de robôs móveis hoje em dia combinam diversas técnicas de \textbf{localização e odometria} para implementar a realimentação necessária pelos sistemas de controle, comparando os resultados entre si. \cite{ginzburg2013indoor} propõe um sistema de localização para robô omnidirecional baseado em odometria (localização relativa) e triangulação ativa de sensores no ambiente (localização absoluta), com fusão de dados para obter o resultado final. \cite{rojas2006holonomic} utiliza a leitura dos \emph{encoders} das rodas e uma câmera externa, enquanto \cite{garcia2015gyro} utiliza apenas um giroscópio e um sensor de distância. \cite{rohrig2010laser}, por outro lado, utiliza medições de distância utilizando sensores laser em AGVs.

\cite{lowcostIMU} mostra que é possível executar um algoritmo de determinação de atitude a partir de uma IMU utilizando TRIAD, filtros de Kalman, covariância de Allen e a plataforma Arduino Uno com razoável precisão, até 40 Hz. \cite{park1996dead} também analisa fusão de dados utilizando um filtro de Kalman indireto para realizar \emph{dead-reckoning} a partir da leitura de \emph{encoders} e um giroscópio. Métodos de localização também são desenvolvidos em outras áreas, como relata \cite{jimenez2009comparison}, que implementa três métodos de localização baseados em INS para trajetórias de pedestres, e conclui que os resultados podem ser melhorados quando há mais qualidade na detecção da orientação. \cite{steinhoff2010pocket} obteve resultados similares na mesma área.

%% SOFTWARE
Dos trabalhos mencionados, poucos entram em detalhes quanto ao \emph{hardware} utilizado. \cite{oubbati2005velocity} utiliza um computador embarcado de 2,6 GHz, uma grande evolução em relação a \cite{feng1989servo}, que utilizava um computador Motorola 68000 com aproximadamente um milésimo da capacidade computacional daquele. \cite{takemura2007development} e \cite{loh2003mechatronics} utilizam computadores externos, envolvendo atrasos na comunicação entre tais computadores e o robô. \cite{lowcostIMU} apresenta um sistema que implementa um filtro de Kalman em um microprocessador Arduino UNO, uam alternativa de baixo custo, e diversos trabalhos mais recentes, como \cite{krinkin2015design}, utilizam especificamente o computador embarcado \textbf{Raspberry Pi} para o processamento e um \textbf{Arduino} para a interface com sensores e atuadores.

% !TEX root = main.tex
% !TEX program = pdflatex


\section{Especificação e montagem do protótipo}
\label{sec:montagem}

Conforme mencionado nas seções anteriores, o robô construído possui três rodas em uma configuração simétrica. Apesar da falta de redundância -- pois se alguma das rodas falhar se perde a holonomicidade --, robôs omnidirecionais com 3 rodas (TOMR) são utilizados com mais frequência por serem mais simples de se implementar, apresentarem custo mais baixo (pois motores e rodas são responsáveis por x\% do custo do projeto, conforme o \hyperref[sec:custo]{Apêndice A}), e uma certa economia de peso.

As rodas utilizadas medem 58 mm de diâmetro, com estrutura em plástico e dez roletes emborrachados, mostrando boa capacidade de carga para os fins de demonstração do projeto. Cada roda é acionada por um motor de corrente contínua com caixa de redução de relação 1:34, com uma velocidade nominal no eixo de saída de 210 rpm. A fixação das rodas no motor foi realizada conforme o Apêndice B \textbf{n sei se vai ter essa frase aqui}. Incluso no motor está um \textit{encoder} de quadratura, que permite a leitura da velocidade da roda e da direção de rotação. Com a relação de redução, se tem que para cada revolução da roda se tem 341.2 pulsos do sensor \cite{motor}.


%Mais info sobre a ponte H: http://linksprite.com/wiki/index.php5?title=DC_Motor_Driver_Breakout_%28L298_Chipset%29#Arduino_Sample_Code

Além da utilização dos \textit{encoders} para implementação da odometria, também foi instalada na estrutura uma bússola, para garantir uma medida absoluta da orientação do robô maior parte do algoritmo de localização. O modelo utilizado é a HMC5883L, em um módulo integrado (breakout board?), que utiliza comunicação pelo protocolo I2C e tem precisão de 2 graus (\cite{HMC5883L}). Devido ao baixo custo dos componentes utilizados e da modularidade do protocolo de comunicação (\cite{semiconductors2000i2c}), também foi adicionada uma unidade de medidas inerciais MPU6050, que possui acelerômetro e giroscópio em torno dos três eixos utilizados (\cite{MPU6050}).

falar da bateria falar da bateria falar da bateria falar da bateria falar da bateria falar da bateria. Ligados à bateria, se tem dois reguladores de tensão MH-MINI-360 CITARDATASHEET, um trabalhando em 5V para alimentar o computador principal e os sensores, e outro com saída configurada em 6 V, para alimentar os motores e os drivers.

O acionamento dos motores se dá por um circuito de pontes H. Há duas destas placas, e cada uma pode acionar dois motores. Assim, se tem a possibilidade de utilizar mais um motor em trabalhos futuros. Talvez eu devesse citar as especificações dessas coisas, certo? Cada driver é alimentado com a tensão regulada de 6 V, e o computador utilizado comanda o chaveamento via PWM.

Power pins
The maximum permitted current draw from the 3.3 V pins is 50 mA.

Maximum permitted current draw from the 5 V pin is the USB input current (usually 1 A) minus any current draw from the rest of the board.[18]

Model A: 1000 mA - 500 mA -> max current draw: 500 mA
Model B: 1000 mA - 700 mA -> max current draw: 300 mA

Todo o processamento é realizado por um \textbf{Raspberry Pi}, um \emph{single board computer}, que utiliza a arquitetura \acrshort{arm} em seu processador, ideal para dispositivos alimentados por baterias por consumir pouca energia e gerar pouco calor. O processador possui quatro núcleos, e um \emph{clock} de 1,2 GHz -- poder computacional equivalente há um computador de mesa comum. O \acrshort{rpi} utiliza um sistema operacional GNU/Linux, e \emph{software} deve ser desenvolvido para ser executado nesta plataforma. Há ainda 40 pinos de \acrshort{gpio} que podem ser utilizados para conectar sensores, atuadores e diversos componentes, e suporte nativo a \acrshort{i2c} (\cite{upton2014raspberry}).

Para unir todos os componentes descritos, se projetou uma estrutura central, como um chassi. Tal estrutura pode ser vista na Figura \ref{fig:chassi}. No centro geométrico da estrutura e na periferia, próximo a uma das rodas, foram feitos dois orifícios que devem acomodar uma caneta hidrográfica cada. Assim, durante a fase de testes, se pode acompanhar graficamente a evolução cinemática do robô. Devido a localização central de uma das canetas, todos os componentes foram instalados na periferia da estrutura. Se tomou ainda o cuidado de instalar o CI de acelerômetro e giroscópio o mais próximo ao centro possível, para que as componentes de aceleração centrípeta dos movimentos com componentes de rotação não influciassem demasiado nos resultados. A IMU poderia ter sido colocada no centro geométrico, e este erro poderia ser introduzido no traço da caneta. No entanto, como a odometria e localização dependem muito mais dos sensores montados nos motores do que da IMU, se preferiu manter a caneta no centro, mantendo o MPU6050 o mais próximo possível. A bússola também foi montada relativamente próxima ao centro do robô, se tomando o cuidado de manter a mesma orientação dos eixos.

\begin{figure}[h]
  \centering
  \includegraphics[width = 0.45\textwidth]{imagens/proto01}
  \caption{Chassi projetado. TROCAR A IMAGEM.}
  \label{fig:chassi}
\end{figure}

O custo de aquisição dos componentes relatados pode ser visto detalhado no hyperref[sec:custo]{Apêndice A}. Cabe ressaltar que todos os itens foram comprados em dobro, para realizar a montagem de dois robôs para futuros trabalhos no LAMECC (Laboratório de Mecatrônia e Controle), e que o chassi foi usinado no próprio laboratório com equipamento próprio, sendo computado apenas o custo do material utilizado.

\cite{siegwart2011introduction}, pg 97: sensores.

\section{Desenvolvimento Teórico}
\label{sec:teorico}

%% MODELAGEM:
%PARK: pg 468

Primeiramente, se definem dois sistemas de coordenadas. O primeiro, $(x_I,y_I)$, é o sistema de coordenadas global, fixo no ambiente. O segundo, $(x_R,y_R)$, está centrado no próprio robô. Ainda se pode definir o ângulo $\theta$ como a orientação do robô -- ou seja, o ângulo entre os dois sistemas de coordenadas. Tal relação pode ser vista na Figura \ref{fig:ref}, e a transformação de um sistema para o outro é descrita na Equação \ref{eq:world_ref}, conforme \cite{siegwart2011introduction} e \cite{ritter2016modelagem}.

\begin{figure}[h]
  \centering
  \includegraphics[width = 0.65\textwidth]{imagens/ref}
  \caption{Sistemas de coordenadas global I e relativo ao centro do robô R.}
  \source{Adaptado de \cite{ritter2016modelagem}}
  \label{fig:ref}
\end{figure}

\begin{equation}
  \begin{pmatrix}
    x_I \\
    y_I \\
    \theta
  \end{pmatrix}
  =
  \begin{pmatrix}
    cos \theta & -sen \theta & 0 \\
    sen\theta  &  cos \theta & 0 \\
    0          & 0          & 1
  \end{pmatrix}
  \begin{pmatrix}
    x_R \\
    y_R \\
    \theta
  \end{pmatrix}
  \label{eq:world_ref}
\end{equation}

O último termo da Equação \ref{eq:world_ref} também pode ser descrito como $q_R$, e o vetor de velocidades $[v_x, v_y, \omega_z]^T$, centrados no sistema de coordenadas do robô, é $\dot{q_R}$. Com o objetivo de mapear a velocidade de giro das rodas $\dot{\phi} = [\dot{\phi}_1, \dot{\phi}_2, \dot{\phi}_3]^T$ às velocidades $\dot{q_R}$, se utiliza a modelagem cinemática apresentada por \cite{siegwart2011introduction}, com as referências apresentadas na Figura \ref{fig:robo_vel}. Na figura,  A mesma modelagem é utilizada por \cite{ritter2016modelagem}, porém com outra sequência e sentido de giro para as rodas.

\begin{figure}[h]
  \centering
  \includegraphics[width = 0.5\textwidth]{imagens/robot_vel4}
  \caption{Vista superior do robô, mostrando as convenções adotadas. As grandezas $v_x$ e $v_y$ estão no sistema de coordenadas do robô.}
  \label{fig:robo_vel}
\end{figure}

Assim, para um robô com 3 rodas dispostas em simetria radial em torno do centro da estrutura, a cinemática direta é dada pela Equação \ref{eq:dk}. Diversos autores utilizam variações da mesma modelagem (\cite{rojas2006holonomic}, \cite{pin1994new}, entre outros). Nas equações apresentadas, $r$ é o raio de cada roda e $R$ o raio do robô (a distância do centro da roda ao centro da estrutura do robô).

\begin{equation}
  \begin{pmatrix}
    v_x \\
    v_y \\
    \omega_z
  \end{pmatrix}
  =
  \frac{r}{3R}
  \begin{pmatrix}
    -\frac{3R}{\sqrt{3}} & 0   & \frac{3R}{\sqrt{3}} \\
    R                    & -2R & R                   \\
    1                    & 1   & R
  \end{pmatrix}
  \begin{pmatrix}
    \dot{\phi_1} \\
    \dot{\phi_2} \\
    \dot{\phi_3}
  \end{pmatrix}.
  \label{eq:dk}
\end{equation}

Também se deseja utilizar a cinemática inversa do modelo, obtida realizando-se a inversão da matriz de transformação apresentada na Equação \ref{eq:dk}, é dada pela Equação \ref{eq:ik}. Nota-se que esta inversão é simplificada no caso do robô com 3 rodas, visto que quando há mais rodas é formada uma matriz $3 \times n$, sendo $n$ o número de rodas, e se deve utilizar uma matriz pseudo-inversa, conforme demonstrado por \cite{rojas2006holonomic}.

\begin{equation}
  \begin{pmatrix}
    \dot{\phi_1} \\
    \dot{\phi_2} \\
    \dot{\phi_3}
  \end{pmatrix}
  =
  \frac{1}{r}
  \begin{pmatrix}
    -\frac{\sqrt{3}}{2} & \frac{1}{2} & R \\
    0                   & -1          & R \\
    \frac{\sqrt{3}}{2}  & \frac{1}{2} & R
  \end{pmatrix}
  \begin{pmatrix}
    v_x \\
    v_y \\
    \omega_z
  \end{pmatrix}
  \label{eq:ik}
\end{equation}

Como pela classificação de \cite{campion1996structural} um \acrshort{tomr} é caracterizado na categoria (3,0), o modelo cinemático das equações \ref{eq:dk} e \ref{eq:ik} é controlável, estável e descreve a posição, orientação e suas derivadas de forma suficiente. % O modelo cinemático da Equação \ref{eq:dk} também é utilizado por \cite{rojas2006holonomic} e \cite{ritter2016modelagem}.

%% ODOMETRIA:
% lynch, pg 492 do pdf

Durante a operação do robô, se torna necessário calcular a posição da estrutura. Para o cálculo da odometria, se utiliza a metodologia mostrada em \cite{lynch2017modern}. Se assume que durante um certo intervalo de tempo $\Delta t$ se tenha velocidades de rotação constantes nas rodas, o que permite considerar $\dot{\phi_i}.\Delta t = \Delta \phi_i$. Considera-se também que a unidade de tempo deste período é arbitrária, e como se deseja integrar no mesmo intervalo posteriormente, se assume um período unitário $\Delta t = 1$. Este procedimento está descrito na Equação \ref{eq:odo}, modificada a partir da Equação \ref{eq:dk}. Na prática, é fácil contar os deslocamentos angulares $\Delta \phi_i$, visto que o número de pulsos por revolução dos \textit{encoders} é determinado.

\begin{equation}
  \begin{pmatrix}
    v_x \\
    v_y \\
    \omega_z
  \end{pmatrix}
  =
  \frac{r}{3R}
  \begin{pmatrix}
    -\frac{3R}{\sqrt{3}} & 0   & \frac{3R}{\sqrt{3}} \\
    R                    & -2R & R                   \\
    1                    & 1   & R
  \end{pmatrix}
  \begin{pmatrix}
    \Delta{\phi_1} \\
    \Delta{\phi_2} \\
    \Delta{\phi_3}
  \end{pmatrix}.
  \label{eq:odo}
\end{equation}

De posse das velocidades da plataforma durante o período de tempo unitário $\Delta t$ -- lembrando que $v_x$, $v_y$ e $\omega_z$ estão no sistema de coordenadas centrado no corpo do robô --, se deve avaliar o deslocamento em relação ao centro do robô na posição anterior. Para o caso em que $\omega_z = 0$, numa trajetória retilínea, se tem simplesmente que $\Delta q_R = \dot{q_R}$.

No entanto, quando houve mudança de orientação no período e consequentemente $\omega_z \neq 0$, se deve levar em consideração os desvios de trajetória causados por essa rotação. Assim, se obtem $\Delta q_R$ de acordo com a Equação \ref{eq:desvio} (\cite{lynch2017modern}).

\begin{equation}
  \Delta q_R
  =
  \begin{pmatrix}
    \Delta x_R \\
    \Delta y_R \\
    \Delta\theta
  \end{pmatrix}
  =
  \begin{pmatrix}
    (v_x sen(\omega_z)) + v_y (cos(\omega_z) - 1)/\omega_z \\
    (v_y sen(\omega_z)) + v_x (1-cos(\omega_z)) / \omega_z \\
    \omega_z
  \end{pmatrix}
  \label{eq:desvio}
\end{equation}

Sendo $k$ o instante antes do período de tempo analisado, para se obter a nova posição $q_I$ do robô no sistema de coordenadas global se deve utilizar a rotação $R(\theta_k)$ apresentada na Equação \ref{eq:world_ref}, e atualizando os valores da última iteração conforme a Equação \ref{eq:new_odo}.

\begin{equation}
  q_{I(k+1)} = q_{I(k)} + \Delta q_I = q_{I(k)} + R(\theta_k) \Delta q_I
  \label{eq:new_odo}
\end{equation}
%\cite{samani2007comprehensive}: Adicionam um modelo de ruído dos encoders à estimativa. TIRAR OU ELABORAR?

%% PLANEJAMENTO DE TRAJETÓRIA:
Para o robô desenvolvido, não há a necessidade de implementar algoritmos complexos de planejamento de trajetória (detecção de obstáculos, caminhos de mínima energia, etc.). Serão abordados caminhos ``ponto a ponto'', que levam de um ponto inicial a um ponto final, ambos em repouso (\cite{lynch2017modern}).

Apesar de ser uma trajetória simples, ainda se podem aplicar considerações para uma melhor operação do sistema. Uma dessas considerações é o chamado \textit{time-scaling} da trajetóra, ou seja, a geração de uma função $s(t)$ que suavize o comportamento do robô por meio de restrições em velocidades e acelerações. Na Figura \ref{fig:poly5} se pode ver uma curva de perfil de velocidade polinomial de quinta ordem, que pode garantir velocidades e acelerações nulas nos pontos de origem e destino.

\begin{figure}[h]
  \centering
  \includegraphics[width = 0.85\textwidth]{imagens/poly5}
  \caption{Deslocamento, velocidade e aceleração durante uma trajetória gerada por polinômio de quinta ordem. Aceleração e velocidade são nulas tanto no ponto de origem quanto no ponto de destino.}
  \source{\cite{lynch2017modern}}
  \label{fig:poly5}
\end{figure}

No entanto, a interpolação de um polinômio a cada cálculo de trajetória é um processo que pode envolver um certo custo computacional elevado, e devido à simplicidade dos componentes utilizados, se julgou que o aumento de suavidade na operação não fosse significativo. Portanto, neste trabalho se optou por utilizar um perfil de velocidade trapezoidal, conforme mostrado na Figura \ref{fig:trap}. Tal perfil é um dos mais comuns em robótica, devido a sua simples implementação. Os limites de aceleração foram definidos na fase de implantação do \textit{software}, de modo a evitar o deslizamento das rodas utilizadas na superfície de testes.

\begin{figure}[h]
  \centering
  \includegraphics[width = 0.63\textwidth]{imagens/trapezoidal}
  \caption{Deslocamento e velocidade durante um deslocamento com perfil de velocidades trapezoidal. Tal perfil foi adotado neste trabalho.}
  \source{\cite{lynch2017modern}}
  \label{fig:trap}
\end{figure}

EM CONSTRUÇÃO DAQUI PRA BAIXO

%% CONTROLE:
CONTROLE:
\cite{rojas2006holonomic}: Sugerem que utilizar um controlador para cada roda é melhor do que para cada grau de liberdade. No nosso caso, é tranquilo pois temos apenas 3 rodas, mantendo o mesmo número de controladores. Devido à realimentação externa lenta, utilizam um preditor no robô. Não entram em detalhes.
motores:
https://www.banggood.com/6V-210RPM-Encoder-Motor-DC-Gear-Motor-with-Mounting-Bracket-and-Wheel-p-1044064.html?p=970719369296201312SG

Given a desired trajectory q d (t), we can adopt the feedforward plus proportional feedback linear controller (11.1) of Chapter 11 to track the trajectory: \cite{lynch2017modern}

\cite{samani2007comprehensive}: Definir os coeficientes dos PIDs é uma novela, pois devemos levar em consideração parâmetros que possuem muita variação, como o coeficiente de atrito do solo, características das baterias, entre outros. Controle deles é bem legal.
%q̇ com (t) = q̇ d (t) + K p ( q̇ d (t) − q(t)),
%(13.11)
%where K p ∈ R 3×3 is positive definite and q(t) is an estimate of the actual con-
%figuration derived from sensors. Then q̇ com (t) can be converted to commanded
%wheel driving velocities u com (t) using Equation (13.7).


%: seguimento de trajetória: malha aberta (3.6.1); Feedback (3.6.2) do livro: It is very similar to the controllers presented in [39, 100]. Others can be found in [8, 52, 53, 137]. Controle com uma matriz de ganhos K para o espaço de estados. Estável e tal. Comentam sobre camadas: planejamento -> decisão -> controlador em tempo real -> hardware.


\cite{lynch2017modern} falar um pouquinho de bodytwist??
FALAR DA BÚSSOLA?

\section{Implementação dos algoritmos}
\label{sec:software}

SEÇÃO EM CONSTRUÇÃO

O código de \cite{ritter2016modelagem} foi desmembrado em módulos, para separar a implementação já realizada da cinemática direta e inversa dos módulos de comunicação com o simulador utilizado. Foram escritos novos módulos para realizar a interface da Raspberry com os motores, sensores e periféricos em geral, e com essa modularidade se pode até fazer uma biblioteca para arduino blabalbalabballab.

COLOCAR DIAGRAMA DAS BIBLIOTECAS

Foi utilizada pigpio \cite{pigpio}.

Além de utilizar os comandos fornecidos por \cite{ritter2016modelagem}, foram implementados os modos de movimentação citados por \cite{loh2003mechatronics}: translação retilínea , translação curvilínea -- ambas sem alteração na orientação --, rotação pura e um caminho combinado de rotação em torno do seu centro e translação retilínia em relação às referências globais.

O \textbf{Raspberry Pi} é um \emph{single board computer}, que utiliza a arquitetura \acrshort{arm} em seu processador, ideal para dispositivos alimentados por baterias por consumir pouca energia e gerar pouco calor. O processador possui quatro núcleos, e um \emph{clock} de 1,2 GHz -- poder computacional equivalente há um computador de mesa comum. O \acrshort{rpi} utiliza um sistema operacional GNU/Linux, e \emph{software} deve ser desenvolvido para ser executado nesta plataforma. Há ainda 40 pinos de \acrshort{gpio} que podem ser utilizados para conectar sensores, atuadores e diversos componentes, e suporte nativo a \acrshort{i2c} (\cite{upton2014raspberry}).


Para a comunicação dos periféricos com este computador, é necessário utilizar algum protocolo de comunicação. O protocolo \textbf{\acrlong{i2c}}, é geralmente utilizado em robôs, csom um grande suporte tanto pela \acrshort{rpi} (\cite{upton2014raspberry}) quanto pelos componentes em geral utilizados (\cite{MPU6050} e a bússola e o arduino se eu usar). Com este protocolo, descrito em \cite{semiconductors2000i2c}, dados podem ser transmitidos a 100 Kbps -- ou 400 Kbps quando utilizado o \emph{fast mode}. São utilizados duas linhas bidirecionais no barramento: SDA para os dados e SCL para os sinais de \emph{clock}. O número de dispositivos conectados ao barramento só depende do limite de capacitância descrito na especificação. Resistores de \emph{pull-up} são necessários para manter a linha em estado lógico alto quando não utilizada, porém estes resistores estão presentes internamente no \emph{Raspberry Pi}, por exemplo.

O \emph{fast mode} é suportado pelo Raspberry Pi 3 B+.

Seguir o perfil de velocidade não é muito fácil, visto que dead zone.

\section{Avaliação experimental}
\label{sec:experimental}

SEÇÃO EM CONSTRUÇÃO

Se notou que o acionamento dos motores depende de alguns fatores. Variando a frequência dos PWMs mudou o q?

O protótipo foi acionado sobre um papel enorme, com duas canetas de cores distintas instaladas nos orifícios destinados a tal. Seguem os resultados:

\section{Resultados}
\label{sec:resultados}

A implantação do sistema descrito nas seções anteriores é, de certa maneira, hierárquica: para que um subssistema de alto-nível funcione, os que estão abaixo dele devem estar funcionando também. Tal conceito pode ser visto na Figura \ref{fig:sistema} Portanto, os testes e validações ocorreram de maneira concomitante ao desenvolvimento do software e montagem da estrutura.

Devido ao fato do computador não ser um sistema 100\% apropriado para tempo real balbalbalbalbabalalbal a tmeporização dos sensores fic preocupante. ASsim, s emediu as leituras obtidas para cada encoder, e se obteve o seguinte, comforme  aFIgura \ref{fig:pos_encoder}.

\begin{figure}[h]
  \centering
  \includegraphics[width = 0.3\textwidth]{imagens/edc}
  \caption{Encoderdoderdoder.}
  \label{fig:pos_encoder}
\end{figure}

Durante os testes de acionamento dos motores, se percebeu uma não linearidade do tipo ``zona-morta'', e foi realizado um ensaio para quantificar o problema. Os resultados de tal ensaio podem ser vistos na Figura \ref{fig:zonamorta}. Foi implementado um controlador proporcional para a velocidade, com ganho bastante baixo, de modo a tornar a resposta do sistema lenta, e foram definidos \emph{setpoints} que propositalmente levassem a saturação dos \textit{drivers} dos atuadores. Na Figura, se pode enxergar a zona morta em torno de $t=$1.5 s, quando a velocidade da roda se eleva repentinamente, e entre $t=$ 4 s e $t=$ 5 s, quando foi realizado o comando de inversão da velocidade da roda, e para um intervalo de valores de acionamento em torno da origem, não há movimento no eixo. Esta não linearidade é oriunda do atrito estático introduzido pelos componentes mecânicos responsáveis pela relação de redução. \textit{tenho mais um gráfico mostrando q dá pra andar bem devagar se já estiver andando no inicio... temos espaço??}

\begin{figure}[h]
  \centering
  \includegraphics[width = \textwidth]{imagens/zonamorta}
  \caption{Análise da não-linearidade do tipo ``zona-morta'' presente nos motores utilizados.}
  \label{fig:zonamorta}
\end{figure}

O ensaio foi realizado simultaneamente com as três rodas, e como todas as leituras apresentaram resultados similares, apenas uma aparece no gráfico.

Uma consequência deste efeito foi verificada no acionamento do robô em trajetórias que exigem baixas velocidades de alguma das rodas. Esta situação pode ser vista na Figura \ref{fig:zm3rodas}. Na figura, se pode ver como os motores que devem operar em velocidades menores entram em operação mais tarde, causando desvios de trajetória. \textit{tenho mais um monte de dados parecidos, para varias velocidades e vários ganhos}

\begin{figure}[h]
  \centering
  \includegraphics[width = \textwidth]{imagens/zm3rodas}
  \caption{Efeito da zona morta no acionamento das 3 rodas com velocidades distintas.}
  \label{fig:zm3rodas}
\end{figure}

(está com velocidades bem baixinhas, se estiver um pouco mais rápido melhora)

\cite{cunha2001zm} propõe um jeito de se resolver isso. Aplicando esse jeito, se conseguiram melhorar bastante os resultaos, veja na FIgura \ref{fig:zm_comp}, logo abaix alskdjhalksjdhlaskdjhal.

\begin{figure}[h]
  \centering
  \includegraphics[width = \textwidth]{imagens/zmcomp}
  \caption{Não linearidade compensada, mostrando o acionamento quase simultâneo de todas as rodas.}
  \label{fig:zm_comp}
\end{figure}

Desvio padrão do tmepo de execução do loop: 16,68573853

Tempo médio a cada chamada do loop de controle, em $\mu$s: 10000,03822

resultados do controle de velocidade

\begin{figure}[h]
  \centering
  \includegraphics[width = \textwidth]{imagens/inst}
  \caption{Resultados de operação aleatória do robô utilizando ganhos arbitrários.}
  \label{fig:inst}
\end{figure}

resultados da calibração da velocidade, com fator de conversão 480 ao invés de 534.
0.1, 0.2, 0.4
\citet{lynch2017modern}: In most robotic applications, higher control update rates are of limited benefit, given
time constants associated with the dynamics of the robot and environment.

resultados da odometria:
Growth of the pose uncertainty for straight-line movement: Note that the uncertainty in y grows much
faster than in the direction of movement. This results from the integration of the uncertainty about the
robot’s orientation. \citep{siegwart2011introduction}

resultados do controle de posição

tempo de duração da bateria

\section{Conclusão e Trabalhos Futuros}
\label{sec:resultados}

O trabalho apresentado neste relatório parcial conta com a introdução, revisão bibliográfica do estado da arte, revisão bibliográfica da fundamentação teórica (incompleta) e alguns comentários a respeito da modelagem que será utilizada. Todo o hardware já foi adquirido, e as próximas do trabalho são:
\begin{itemize}
  \item{fabricação do chassi do protótipo;}
  \item{finalização da revisão bibliográfica;}
  \item{modelagem do sistema;}
  \item{projeto e implantação dos controladores e métodos de odometria;}
  \item{montagem;}
  \item{experimentos e avaliações dos mesmos;}
  \item{adequação de detalhes do formato do trabalho, para que se enquadre ao modelo.}
\end{itemize}

A escrita do trabalho é realizada em paralelo ao desenvolvimento do mesmo, para melhor controle e documentação do projeto.


%% Bibliografia
\bibliography{biblio}

%% Anexos e Apêndices
\apendices
  \subsection{Custo dos componentes utilizados}
\label{sec:custo}

Há no mercado uma variada gama de componentes a serem utilizados em projetos robóticos. Para o projeto em questão, se utilizaram componentes que mostrassem um preço de mercado competitivo e grande disponibilidade. Dessa forma, se pode manter o projeto viável, mesmo que por vezes sacrificando um possível incremento de desempenho que se daria ao utilizar um componente mais robusto, por exemplo.

Na Tabela \ref{tab:custo} se pode ver a lista de componentes adquirida e os respectivos custos. Nota-se que no caso das 3 rodas há -- integrado ao valor apresentado -- as taxas de importação e conversão de moedas, visto que esses componentes foram importados dos Estados Unidos. Também é importante mencionar que o chassi foi produzido sem custo BLABLABLAB LAMECC.

\begin{table}
  \caption{Custo dos componentes utilizados no projeto.}
  \begin{tabular}{||l r c r||}
     \hline
     \textbf{Item:}   & \textbf{Valor por unidade:} & \textbf{Quantidade:} & \textbf{Valor total:} \\ \hline\hline
     Omniwheel        & R\$ 46,76                   &  3                   & R\$ 140,29            \\ \hline
     Motor c/ encoder & R\$ 119,00                  &  3                   & R\$ 357,00            \\ \hline
     Driver           & R\$ 15,00                   &  2                   & R\$ 30,00             \\ \hline
     Raspberry Pi     & R\$ 148,00                  &  1                   & R\$ 148,00            \\ \hline
     microSD 16GB     & R\$ 34,00                   &  1                   & R\$ 34,00             \\ \hline
     Arduino Mega     & R\$ 40,00                   &  1                   & R\$ 40,00             \\ \hline
     IMU MPU6050      & R\$ 9,00                    &  1                   & R\$ 9,00              \\ \hline
     Magnetômetro HMC5883 & R\$ 12,80               &  1                   & R\$ 12,80             \\ \hline
     Chassi           & R\$ 5,00                    &  1                   & R\$ 5,00              \\ \hline
     Bateria          & R\$ 100,00                  &  1                   & R\$ 100,00            \\ \hline
     Reguladores de Tensão & R\$ 4,98               &  4                   & R\$ 19,96             \\ \hline
     Diversos         & R\$ 30,00                   &  X                   & R\$ 30,00             \\ \hline\hline
                      &                             & \textbf{Custo Total:} & R\$ 926,05           \\ \hline
  \end{tabular}
  \label{tab:custo}
\end{table}

Numa escala de preços em robótica, se percebe que (COMPARAR OS PREÇOS DE OUTROS PROTÓTIPOS??)


\end{document}
